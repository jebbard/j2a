%-----------------------------------------------------------------------------------------------
%		Introduction
%-----------------------------------------------------------------------------------------------

\chapter{Einleitung}
\label{sec:Einleitung}

Dieses Dokument stellt eine Kurzspezifikation eines Hilfswerkzeugs f�r die Entwicklung einer JavaEE-Anwendung dar. Es handelt sich dabei um einen Generator, der aus einer gegebenen Java-Klasse, die einer Standard-JPA-Entit�t entspricht, auf Basis von Templates andere Java-Klassen generieren kann, z.B. "`Transport-Objekte"' (TOs) oder "`Data Access Objects"' (DAOs). Arbeitstitel des Werkzeuges ist \emph{"`TO from JPA entity generator"'}.

Grundlegende Anforderungen an das Werkzeug:
\begin{itemize}
	\item Alle Standard-JPA-2.1-konformen Entit�ten k�nnen als Input verwendet werden; die zu verwendenden Entit�ten werden �ber Quellpfad und dort befindliche Quellpackages sowie weitere Einschr�nkungsm�glichkeiten angegeben.
	\item Alle JPA-relevanten Informationen werden aus der Entit�t geparst und k�nnen dann �ber \emph{Template-Variablen} in beliebig vielen \emph{Templates} verwendet werden.
	\item Ein Template ist eine Schablone einer durch den Generator zu erzeugenden Java-Klasse, die im Rahmen des Generierungsprozesses auf Basis der gelesenen Entit�ten expandiert wird. Aus einer Schablone k�nnen dabei N generierte Java-Klassen entstehen.
\end{itemize}

Ausschl�sse der Initialversion \TOGenVersion{}:
\begin{itemize}
	\item Es werden nur annotierte Entit�ten unterst�tzt, XML-Mappings k�nnen nicht ausgewertet werden.
	\item Mapped superclasses oder Entit�ts-Superklassen werden bei Generierung nicht ber�cksichtigt.
	\item Der Generator navigiert Beziehungen zwischen Entit�ten nicht explizit. Ist eine in Beziehung zur Quellentit�t stehende Entit�t nicht in den angegebenen Quellpaketen der Generierung enthalten, so wird der Typ der referenzierten Fremd-Entit�t in die generierten Klassen �bernommen. F�r die Fremd-Entit�t erfolgt dann keine Generierung.
	\item Spezielle Annotationen von Implementierungen (Hibernate, OpenJPA etc.) werden ignoriert.
	\item Der Generator erzeugt immer komplett neue Klassen, ein Mergen in bereits generierte Klassen ist nicht m�glich.
	\item Annotationen an Accessor-Methoden statt an Attributen in einer Entit�t werden nicht unterst�tzt, die JPA-Annotationen m�ssen ausschlie�lich an Feldern verwendet werden.
	\item Eingebettete Klassen werden momentan nicht geparst.
	\item Eine grundlegende Erweiterbarkeit durch eigene Template-Generierungsregeln wird aktuell noch nicht unterst�tzt.
\end{itemize}